%% bare_conf.tex
%% V1.4b
%% 2015/08/26
%% by Michael Shell
%% See:
%% http://www.michaelshell.org/
%% for current contact information.
%%
%% This is a skeleton file demonstrating the use of IEEEtran.cls
%% (requires IEEEtran.cls version 1.8b or later) with an IEEE
%% conference paper.
%%
%% Support sites:
%% http://www.michaelshell.org/tex/ieeetran/
%% http://www.ctan.org/pkg/ieeetran
%% and
%% http://www.ieee.org/

%%*************************************************************************
%% Legal Notice:
%% This code is offered as-is without any warranty either expressed or
%% implied; without even the implied warranty of MERCHANTABILITY or
%% FITNESS FOR A PARTICULAR PURPOSE! 
%% User assumes all risk.
%% In no event shall the IEEE or any contributor to this code be liable for
%% any damages or losses, including, but not limited to, incidental,
%% consequential, or any other damages, resulting from the use or misuse
%% of any information contained here.
%%
%% All comments are the opinions of their respective authors and are not
%% necessarily endorsed by the IEEE.
%%
%% This work is distributed under the LaTeX Project Public License (LPPL)
%% ( http://www.latex-project.org/ ) version 1.3, and may be freely used,
%% distributed and modified. A copy of the LPPL, version 1.3, is included
%% in the base LaTeX documentation of all distributions of LaTeX released
%% 2003/12/01 or later.
%% Retain all contribution notices and credits.
%% ** Modified files should be clearly indicated as such, including  **
%% ** renaming them and changing author support contact information. **
%%*************************************************************************


% *** Authors should verify (and, if needed, correct) their LaTeX system  ***
% *** with the testflow diagnostic prior to trusting their LaTeX platform ***
% *** with production work. The IEEE's font choices and paper sizes can   ***
% *** trigger bugs that do not appear when using other class files.       ***                          ***
% The testflow support page is at:
% http://www.michaelshell.org/tex/testflow/



\documentclass[conference]{IEEEtran}
% Some Computer Society conferences also require the compsoc mode option,
% but others use the standard conference format.
%
% If IEEEtran.cls has not been installed into the LaTeX system files,
% manually specify the path to it like:
% \documentclass[conference]{../sty/IEEEtran}





% Some very useful LaTeX packages include:
% (uncomment the ones you want to load)


% *** MISC UTILITY PACKAGES ***
%
%\usepackage{ifpdf}
% Heiko Oberdiek's ifpdf.sty is very useful if you need conditional
% compilation based on whether the output is pdf or dvi.
% usage:
% \ifpdf
%   % pdf code
% \else
%   % dvi code
% \fi
% The latest version of ifpdf.sty can be obtained from:
% http://www.ctan.org/pkg/ifpdf
% Also, note that IEEEtran.cls V1.7 and later provides a builtin
% \ifCLASSINFOpdf conditional that works the same way.
% When switching from latex to pdflatex and vice-versa, the compiler may
% have to be run twice to clear warning/error messages.






% *** CITATION PACKAGES ***
%
%\usepackage{cite}
% cite.sty was written by Donald Arseneau
% V1.6 and later of IEEEtran pre-defines the format of the cite.sty package
% \cite{} output to follow that of the IEEE. Loading the cite package will
% result in citation numbers being automatically sorted and properly
% "compressed/ranged". e.g., [1], [9], [2], [7], [5], [6] without using
% cite.sty will become [1], [2], [5]--[7], [9] using cite.sty. cite.sty's
% \cite will automatically add leading space, if needed. Use cite.sty's
% noadjust option (cite.sty V3.8 and later) if you want to turn this off
% such as if a citation ever needs to be enclosed in parenthesis.
% cite.sty is already installed on most LaTeX systems. Be sure and use
% version 5.0 (2009-03-20) and later if using hyperref.sty.
% The latest version can be obtained at:
% http://www.ctan.org/pkg/cite
% The documentation is contained in the cite.sty file itself.






% *** GRAPHICS RELATED PACKAGES ***
%
\ifCLASSINFOpdf
  % \usepackage[pdftex]{graphicx}
  % declare the path(s) where your graphic files are
  % \graphicspath{{../pdf/}{../jpeg/}}
  % and their extensions so you won't have to specify these with
  % every instance of \includegraphics
  % \DeclareGraphicsExtensions{.pdf,.jpeg,.png}
\else
  % or other class option (dvipsone, dvipdf, if not using dvips). graphicx
  % will default to the driver specified in the system graphics.cfg if no
  % driver is specified.
  % \usepackage[dvips]{graphicx}
  % declare the path(s) where your graphic files are
  % \graphicspath{{../eps/}}
  % and their extensions so you won't have to specify these with
  % every instance of \includegraphics
  % \DeclareGraphicsExtensions{.eps}
\fi
% graphicx was written by David Carlisle and Sebastian Rahtz. It is
% required if you want graphics, photos, etc. graphicx.sty is already
% installed on most LaTeX systems. The latest version and documentation
% can be obtained at: 
% http://www.ctan.org/pkg/graphicx
% Another good source of documentation is "Using Imported Graphics in
% LaTeX2e" by Keith Reckdahl which can be found at:
% http://www.ctan.org/pkg/epslatex
%
% latex, and pdflatex in dvi mode, support graphics in encapsulated
% postscript (.eps) format. pdflatex in pdf mode supports graphics
% in .pdf, .jpeg, .png and .mps (metapost) formats. Users should ensure
% that all non-photo figures use a vector format (.eps, .pdf, .mps) and
% not a bitmapped formats (.jpeg, .png). The IEEE frowns on bitmapped formats
% which can result in "jaggedy"/blurry rendering of lines and letters as
% well as large increases in file sizes.
%
% You can find documentation about the pdfTeX application at:
% http://www.tug.org/applications/pdftex





% *** MATH PACKAGES ***
%
%\usepackage{amsmath}
% A popular package from the American Mathematical Society that provides
% many useful and powerful commands for dealing with mathematics.
%
% Note that the amsmath package sets \interdisplaylinepenalty to 10000
% thus preventing page breaks from occurring within multiline equations. Use:
%\interdisplaylinepenalty=2500
% after loading amsmath to restore such page breaks as IEEEtran.cls normally
% does. amsmath.sty is already installed on most LaTeX systems. The latest
% version and documentation can be obtained at:
% http://www.ctan.org/pkg/amsmath





% *** SPECIALIZED LIST PACKAGES ***
%
%\usepackage{algorithmic}
% algorithmic.sty was written by Peter Williams and Rogerio Brito.
% This package provides an algorithmic environment fo describing algorithms.
% You can use the algorithmic environment in-text or within a figure
% environment to provide for a floating algorithm. Do NOT use the algorithm
% floating environment provided by algorithm.sty (by the same authors) or
% algorithm2e.sty (by Christophe Fiorio) as the IEEE does not use dedicated
% algorithm float types and packages that provide these will not provide
% correct IEEE style captions. The latest version and documentation of
% algorithmic.sty can be obtained at:
% http://www.ctan.org/pkg/algorithms
% Also of interest may be the (relatively newer and more customizable)
% algorithmicx.sty package by Szasz Janos:
% http://www.ctan.org/pkg/algorithmicx




% *** ALIGNMENT PACKAGES ***
%
%\usepackage{array}
% Frank Mittelbach's and David Carlisle's array.sty patches and improves
% the standard LaTeX2e array and tabular environments to provide better
% appearance and additional user controls. As the default LaTeX2e table
% generation code is lacking to the point of almost being broken with
% respect to the quality of the end results, all users are strongly
% advised to use an enhanced (at the very least that provided by array.sty)
% set of table tools. array.sty is already installed on most systems. The
% latest version and documentation can be obtained at:
% http://www.ctan.org/pkg/array


% IEEEtran contains the IEEEeqnarray family of commands that can be used to
% generate multiline equations as well as matrices, tables, etc., of high
% quality.




% *** SUBFIGURE PACKAGES ***
%\ifCLASSOPTIONcompsoc
%  \usepackage[caption=false,font=normalsize,labelfont=sf,textfont=sf]{subfig}
%\else
%  \usepackage[caption=false,font=footnotesize]{subfig}
%\fi
% subfig.sty, written by Steven Douglas Cochran, is the modern replacement
% for subfigure.sty, the latter of which is no longer maintained and is
% incompatible with some LaTeX packages including fixltx2e. However,
% subfig.sty requires and automatically loads Axel Sommerfeldt's caption.sty
% which will override IEEEtran.cls' handling of captions and this will result
% in non-IEEE style figure/table captions. To prevent this problem, be sure
% and invoke subfig.sty's "caption=false" package option (available since
% subfig.sty version 1.3, 2005/06/28) as this is will preserve IEEEtran.cls
% handling of captions.
% Note that the Computer Society format requires a larger sans serif font
% than the serif footnote size font used in traditional IEEE formatting
% and thus the need to invoke different subfig.sty package options depending
% on whether compsoc mode has been enabled.
%
% The latest version and documentation of subfig.sty can be obtained at:
% http://www.ctan.org/pkg/subfig




% *** FLOAT PACKAGES ***
%
%\usepackage{fixltx2e}
% fixltx2e, the successor to the earlier fix2col.sty, was written by
% Frank Mittelbach and David Carlisle. This package corrects a few problems
% in the LaTeX2e kernel, the most notable of which is that in current
% LaTeX2e releases, the ordering of single and double column floats is not
% guaranteed to be preserved. Thus, an unpatched LaTeX2e can allow a
% single column figure to be placed prior to an earlier double column
% figure.
% Be aware that LaTeX2e kernels dated 2015 and later have fixltx2e.sty's
% corrections already built into the system in which case a warning will
% be issued if an attempt is made to load fixltx2e.sty as it is no longer
% needed.
% The latest version and documentation can be found at:
% http://www.ctan.org/pkg/fixltx2e


%\usepackage{stfloats}
% stfloats.sty was written by Sigitas Tolusis. This package gives LaTeX2e
% the ability to do double column floats at the bottom of the page as well
% as the top. (e.g., "\begin{figure*}[!b]" is not normally possible in
% LaTeX2e). It also provides a command:
%\fnbelowfloat
% to enable the placement of footnotes below bottom floats (the standard
% LaTeX2e kernel puts them above bottom floats). This is an invasive package
% which rewrites many portions of the LaTeX2e float routines. It may not work
% with other packages that modify the LaTeX2e float routines. The latest
% version and documentation can be obtained at:
% http://www.ctan.org/pkg/stfloats
% Do not use the stfloats baselinefloat ability as the IEEE does not allow
% \baselineskip to stretch. Authors submitting work to the IEEE should note
% that the IEEE rarely uses double column equations and that authors should try
% to avoid such use. Do not be tempted to use the cuted.sty or midfloat.sty
% packages (also by Sigitas Tolusis) as the IEEE does not format its papers in
% such ways.
% Do not attempt to use stfloats with fixltx2e as they are incompatible.
% Instead, use Morten Hogholm'a dblfloatfix which combines the features
% of both fixltx2e and stfloats:
%
% \usepackage{dblfloatfix}
% The latest version can be found at:
% http://www.ctan.org/pkg/dblfloatfix




% *** PDF, URL AND HYPERLINK PACKAGES ***
%
%\usepackage{url}
% url.sty was written by Donald Arseneau. It provides better support for
% handling and breaking URLs. url.sty is already installed on most LaTeX
% systems. The latest version and documentation can be obtained at:
% http://www.ctan.org/pkg/url
% Basically, \url{my_url_here}.




% *** Do not adjust lengths that control margins, column widths, etc. ***
% *** Do not use packages that alter fonts (such as pslatex).         ***
% There should be no need to do such things with IEEEtran.cls V1.6 and later.
% (Unless specifically asked to do so by the journal or conference you plan
% to submit to, of course. )


% correct bad hyphenation here
\hyphenation{op-tical net-works semi-conduc-tor}


\usepackage[brazilian]{babel}
\usepackage[utf8]{inputenc}

\begin{document}
%
% paper title
% Titles are generally capitalized except for words such as a, an, and, as,
% at, but, by, for, in, nor, of, on, or, the, to and up, which are usually
% not capitalized unless they are the first or last word of the title.
% Linebreaks \\ can be used within to get better formatting as desired.
% Do not put math or special symbols in the title.
\title{Planejamento Regional Adaptativo em Sistemas Self-Adaptive de Larga Escala}


% author names and affiliations
% use a multiple column layout for up to three different
% affiliations
\author{

	\IEEEauthorblockN{Eudes S. Andrade e Sandro S. Andrade}
	\IEEEauthorblockA{GSORT -- Grupo de Pesquisa em Sistemas Distribuídos, Otimização, Redes e Tempo Real\\
		IFBA -- Instituto Federal de Educação, Ciência e Tecnologia da Bahia\\
		Av. Araújo Pinho, nº 39 -- Canela -- CEP: 40.110-150 -- Salvador-BA\\
	Email: \{eudes.andrade, sandro.andrade\}@ifba.edu.br}
}

% conference papers do not typically use \thanks and this command
% is locked out in conference mode. If really needed, such as for
% the acknowledgment of grants, issue a \IEEEoverridecommandlockouts
% after \documentclass

% for over three affiliations, or if they all won't fit within the width
% of the page, use this alternative format:
% 
%\author{\IEEEauthorblockN{Michael Shell\IEEEauthorrefmark{1},
%Homer Simpson\IEEEauthorrefmark{2},
%James Kirk\IEEEauthorrefmark{3}, 
%Montgomery Scott\IEEEauthorrefmark{3} and
%Eldon Tyrell\IEEEauthorrefmark{4}}
%\IEEEauthorblockA{\IEEEauthorrefmark{1}School of Electrical and Computer Engineering\\
%Georgia Institute of Technology,
%Atlanta, Georgia 30332--0250\\ Email: see http://www.michaelshell.org/contact.html}
%\IEEEauthorblockA{\IEEEauthorrefmark{2}Twentieth Century Fox, Springfield, USA\\
%Email: homer@thesimpsons.com}
%\IEEEauthorblockA{\IEEEauthorrefmark{3}Starfleet Academy, San Francisco, California 96678-2391\\
%Telephone: (800) 555--1212, Fax: (888) 555--1212}
%\IEEEauthorblockA{\IEEEauthorrefmark{4}Tyrell Inc., 123 Replicant Street, Los Angeles, California 90210--4321}}




% use for special paper notices
%\IEEEspecialpapernotice{(Invited Paper)}




% make the title area
\maketitle

% As a general rule, do not put math, special symbols or citations
% in the abstract
\begin{abstract}
Construir sistemas computacionais dotados de alguma capacidade de autogerencimento ou auto-adaptação é uma tarefa complexa. Tendo em vista que os sistemas distribuídos têm se tornado cada vez maiores e mais complexos, as soluções para construção de sistemas autogerenciados centralizados não têm se mostrado efetivas. Apesar de algumas propostas para o projeto de sistemas autogerenciados distribuídos estarem disponíveis na literatura, todas elas apresentam como fator limitante a adoção de topologias estáticas no projeto do sistema a ser adaptado.\\Este trabalho tem como objetivo o projeto, implementação e avaliação de um modelo adaptativo para auto gerenciamento em sistemas \textit{self-adaptive} de larga escala. Através dessa abordagem espera-se obter desempenho satisfatório do controle tanto em situações onde o sistema/ambiente sugerem a adoção de topologias mais centralizadas, até situações onde padrões completamente descentralizados apresentam melhor \textit{trade-off}.
\end{abstract}

% no keywords




% For peer review papers, you can put extra information on the cover
% page as needed:
% \ifCLASSOPTIONpeerreview
% \begin{center} \bfseries EDICS Category: 3-BBND \end{center}
% \fi
%
% For peerreview papers, this IEEEtran command inserts a page break and
% creates the second title. It will be ignored for other modes.
\IEEEpeerreviewmaketitle



\section{Introdução}
% no \IEEEPARstart

As diversas demandas envolvidas no desenvolvimento de sistemas computacionais modernos têm modificado a forma como tais aplicações são projetadas, desenvolvidas e avaliadas. Dentre tais demandas, destacam-se: a introdução de computadores multiprocessados, a abundância de paralelismo em soluções altamente distribuídas, a facilidade de integração e a capacidade de autogerenciamento em ambientes incertos \cite{DBLP:journals/ijhpca/Heroux09,DBLP:conf/icse/Northrop13,Gamell:2014:EAO:2683593.2683691}. Muitas dessas novas demandas são consequências do aumento de poder computacional apresentado pelos computadores modernos e pela crescente velocidade e confiabilidade das tecnologias de redes de comunicação. Essa transição dos sistemas da geração \textit{petascale} para o mundo dos sistemas \textit{exascale} \cite{DBLP:conf/ipps/Yelick08} implica investimento em diversas áreas de pesquisa da Ciência da Computação.

Apesar de todas as estratégias disponibilizadas pela Engenharia de \textit{Software} para gerenciamento da complexidade gerada por esses novos desafios, acredita-se que a capacidade humana de compreensão e manipulação de artefatos de \textit{software} se mostrará como um fator limitante em um futuro próximo \cite{DBLP:journals/csur/HuebscherM08}. A complexidade dos produtos de \textit{software} irá crescer continuamente até um ponto onde as tecnologias mais efetivas e os profissionais mais bem formados não serão capazes de construir soluções satisfatórias. \cite{DBLP:journals/computer/KephartC03}. Para ilustrar essa crescente complexidade, pode-se citar a grande quantidade de parâmetros para configuração dos bancos de dados atuais, as diversas possibilidades de \textit{tuning} dos servidores \textit{web} e as constantes adaptações sofridas pelos nós que formam as plataformas de \textit{cloud computing} modernas.

Tendo em vista que os maiores desafios expostos decorrem de requisitos não-funcionais -- e que estes têm sido historicamente tratados através de decisões de projeto -- linhas de pesquisa modernas apontam para uma nova abordagem: a transferência de determinadas decisões de projeto para \textit{runtime}. Para tanto, os sistemas computacionais deverão estar dotados de alguma capacidade de autogerencimento ou auto-adaptação \cite{DBLP:journals/csur/HuebscherM08,DBLP:journals/computer/KephartC03,DBLP:journals/taas/SalehieT09}.

De acordo com a DARPA (\textit{Defense Advanced Research Projects Agency}) um sistema \textit{self-adaptive} é aquele que avalia o seu próprio comportamento e o modifica quando a avaliação indica que: i) o seu propósito principal não está sendo efetivamente cumprido; ou ii) uma melhor funcionalidade e/ou desempenho pode ser alcançado \cite{DARPA-selfadaptive}. A auto-adaptação é uma estratégia utilizada para os mais diversos fins, desde a otimização de performance em sistemas que operam sobre ambientes que mudam constantemente até situações mais extremas onde sistemas conseguem se recuperar após a falha de parte dos seus componentes \cite{DBLP:conf/dagstuhl/WeynsSGMMPWAGG10}. Apesar das diversas aplicações, temos como característica comum a todos os projetos que fazem uso de auto-adaptação a necessidade de migrar determinadas decisões -- que anteriormente eram tomadas em tempo de projeto -- para \textit{runtime}. A motivação para a tomada de decisão tardia (em \textit{runtime}) é justificada uma vez que o arquiteto não mais precisará se comprometer de forma prematura com determinadas decisões de projeto. Decisões estas que, frequentemente, quando tomadas de forma prematura, podem não se mostrar interessantes em \textit{runtime}. Não se mostram interessantes sobretudo por conta da imprevisibilidade/dinamismo do ambiente de execução bem como das características funcionais do próprio sistema. Sendo assim, ao dotar o software de capacidades auto-adaptativas, busca-se minimizar a quantidade de decisões \textit{off-line} -- diminuindo o comprometimento do arquiteto com os \textit{trade-offs} impostos por determinados atributos de qualidade -- rumo à visão utópica de \textit{software} auto-construtivo [4].

Em linhas gerais, soluções auto-adaptativas são bastante efetivas em ambientes onde existe uma grande complexidade no espaço de soluções, acentuado dinamismo nos dados, grande demanda por serviços ou imprevisibilidade nos ambientes de execução. Os desafios acima expostos têm sido objeto de pesquisa em diferentes áreas da Ciência da Computação, dentre elas a Computação Distribuída. São exemplos de soluções distribuídas que utilizam recurso de auto-adaptação os servidores \textit{web} auto-otimizáveis e as plataformas elásticas para \textit{cloud computing}.

Ao mesmo tempo, os sistemas distribuídos têm se tornado cada vez maiores e mais complexos \cite{DBLP:conf/icse/Northrop13}. A adoção de ambientes distribuídos com foco em alta disponibilidade e alto desempenho fez que com que \textit{clusters} formados por mais de 40.000 máquinas se tornassem uma realidade (ex: ambientes de \textit{cloud computing} baseados no Hadoop \cite{white2009hadoop}). A constatação de que soluções centralizadas -- anteriormente aplicadas com sucesso -- não são mais vistas como efetivas atualmente, trouxe algumas consequências. Assim como nos sistemas não-adaptativos, soluções de adaptação centralizadas passaram a não ser mais efetivas. Não eram mais efetivas por conta da imprevisibilidade imposta pelos ambientes distribuídos cada vez maiores e mais complexo. Não eram mais efetivas sobretudo por conta da crescente necessidade de controle em componentes específicos da solução distribuída para que ótimos globais pudessem ser alcançados. De acordo com [8], os sistemas auto-adaptativos que eram originalmente implementados utilizando uma estrutura de controle centralizada, deram espaço à projetos de controle potencialmente distribuídos.

Muitas arquiteturas para autogerenciamento estão disponíveis atualmente no mercado. Ainda assim, a maioria delas apresenta como restrição o suporte, apenas, a sistemas centralizados. Algumas poucas arquiteturas implementam suporte a ambientes descentralizados e as poucas que o fazem não apresentam bom desempenho em ambientes altamente dinâmicos \cite{DBLP:conf/dagstuhl/WeynsSGMMPWAGG10}. A dificuldade para se alcançar autogerenciamento em ambientes descentralizados decorre do inerente \textit{trade-off} existente em sistemas distribuídos: qualidade da solução versus sua previsibilidade temporal. Ou seja, quanto maior o desempenho do sistema, menor a capacidade dele funcionar de maneira satisfatória em situações adversas (seja do ambiente ou da própria operação do sistema). Pode-se ilustrar esse \textit{trade-off} analisando \textit{clusters} para suporte a \textit{cloud computing}. Em tais situações, quanto maior o \textit{overhead} previsto menor será a capacidade do sistema se recuperar em caso de mudanças no ambiente. Por outro lado, quanto maior a folga observada no sistema maior será o consumo de recursos. O objetivo das arquiteturas para autogerenciamento descentralizado em tais \textit{clusters} é reagir ao \textit{trade-off} que envolve a capacidade de atendimento da \textit{cloud} frente às diversas configurações de ambiente apresentadas ao longo do tempo.

A adoção de uma determinada topologia ao modelar o sistema responsável pela adaptação em detrimento de outra é função dos requisitos do sistema a ser monitorado e do ambiente ao qual este estará exposto. Em [8] seis padrões para modelagem de autogerenciamento descentralizado são propostos. Além dos modelos de comunicação a serem realizados entre os diversos componentes do controle, são apresentados os atributos de qualidade inseridos por cada padrão, bem como as restrições impostas por tais decisões. Ainda assim, [8] não considera a possibilidade de adaptação dinâmica das topologias apresentadas. Dessa forma, apesar de os sistemas monitorados serem reconfigurados em \textit{runtime}, os sistemas responsáveis pela adaptação não tem nenhuma capacidade autoadaptativa.

Este trabalho tem como objetivo o projeto e avaliação de um modelo adaptativo para autogerenciamento de \textit{Sistemas Self-Adaptive} (SSA) de larga escala. Esse modelo prevê a reconfiguração dinâmica da topologia e forma de comunicação entre os múltiplos componentes locais do controle. Através dessa abordagem, decisões de projeto acerca da topologia de controle a ser adotada serão levadas para \textit{runtime}. Com isso, espera-se obter desempenho satisfatório do controle tanto em situações onde o sistema monitorado/ambiente sugerem a adoção de topologias mais centralizadas, até situações onde padrões completamente descentralizados apresentam melhor \textit{trade-off}.

O mecanismo de reconfiguração dinâmica da arquitetura de controle proposta será modelado e avaliado via simulação de eventos discretos. Trabalhos futuros confrontarão os resultados da simulação com a execução de protótipos reais que implementam o mecanismo de reconfiguração dinâmica.

Esse trabalho está organizado como segue. A seção 2 apresenta o referencial teórico do trabalho. A seção 3 traz a proposta do mecanismo de reconfiguração dinâmica da arquitetura de controle. A seção 4 apresenta os métodos e materiais ampregados na proposta e validação. A seção 5 apresenta os resultados.  A seção 6 traz os trabalhos relacionados. Por fim, a seção 7 apresenta as conclusões e sugestões de trabalhos futuros.


\section{Referencial Teórico}

Serão apresentadas nessa sessão definições acerca de duas matérias: Sistemas \textit{Self-Adaptive} e Padrões de Projeto para Sistemas \textit{Self-Adaptive}. Em um primeiro momento será discutida a motivação para a adoção de soluções auto-adaptivas e seus fundamentos básicos. Na sessão seguinte serão apresentados seis padrões de projeto para controle descentralizado em sistemas auto-adaptativos. Após o detalhamento dos atributos de qualidade inseridos por cada um dos seis padrões serão trazidas à luz as restrições impostas por tais decisões.

\subsection{Sistemas Self-Adaptive}

A auto-adaptação é uma estratégia de projeto utilizada para os mais diversos fins. Ainda assim, dentre as diversas características comuns a todos os projetos que lançam mão de auto-adaptação, uma se destaca: a necessidade de migrar determinadas decisões que anteriormente eram tomadas em tempo de projeto, para \textit{runtime}. Através desse artíficio, busca-se minimizar possíveis inflexibilidades/ineficiências decorrentes de compromentimentos prévios do arquiteto de \textit{software} com determinados atributos de qualidade. Sabe-se que a ocorrência de situações não previstas pelo arquiteto de \textit{sorftware} decorre, sobretudo, do dinamismo dos ambientes de execução e dos casos de uso da aplicação. Sendo assim, quanto menor a quantidade de premissas e compromentementos prévios realizados pelo arquiteto, maior será a capacidade de evolução e flexibilização das arquiteturas de \textit{software} frente aos diversos cenários de uso.

Para ilustrar tal situação, podemos considerar um cenário onde uma solução não-adaptativa foi escolhida em detrimento de um algorítmo adaptativo. Algoritmos não-daptativos demandam uma série de comprometimentos prévios com planos de execução pré-definidos. O arquiteto pode, por exemplo, ser obrigado a fazer opção por uma determinada solução para \textit{caching} e paralelismo em detrimento de outras. Apesar de essa decisão do arquiteto parecer inicialmente acertada, ela pode implicar perda de eficiência caso os padrões de acesso e/ou a natureza dos dados mudem.\\
Por outro lado, com a adoção de algorítmos adaptativos, o arquiteto de \textit{software} poderá não mais tomar determinadas decisões em tempo de projeto, protelando-as para \textit{runtime}. No exemplo citado, as decisões acerca do melhor algoritmo de \textit{cache} a ser adotado bem como do mecanismo de paralelismo ótimo serão tomadas em \textit{runtime}. Por serem tomadas em \textit{runtime}, essas decisões irão considerar as peculiaridades dinâmicas em dado instante, implicando melhor performance do sistema nos seus diferentes cenários de uso.\\
Definir mecanismos onde o próprio sistema computacional monitore suas condições em dado instante e se reconfigure/replaneje resolve não apenas as incertezas introduzidas pelos ambientes dinâmicos. Em situações onde a realização de \textit{tuning} de sistemas computacionais envolve um alto e dependente número de parâmetros (o que eventualmente pode encontrar como limite as habilidades/capacidades humanas), soluções adaptativas mostram-se efetivas[referencia]. De uma maneira geral, sistemas autogerenciaveis apresentam como requisitos algumas das seguintes características: alta complexidade do espaço de problema e espaço da solução, dinamismo nos dados, dinamismo nas demandas por serviços, dinamismo nos ambientes de execução e características funcionais auto-adaptativas.

\subsubsection{Fundamentos}

Diversas definições de sistemas auto-adaptativos podem ser encontrados na literatura [18, 62, 145, 194, 267]. Uma das definições mais recorrentes é a proposta pela DARPA (\textit{Defense Advanced Research Projects Agency}), em 1997.

“Um sistema \textit{self-adaptive} é aquele que avalia o seu próprio comportamento e o modifica quando a avaliação indica que: i) o seu propósito principal não está sendo efetivamente cumprido; ou ii) uma melhor funcionalidade e/ou desempenho pode ser alcançado.”

Embora essa definição capture em alguma medida as principais motivações para a adoção de soluções auto-adaptativas, uma análise mais detalhada dos elementos que a caracterizam se faz necessário. Para tanto, \textit{Salehie e Tahvildari} [267] utilizam a abordagem do poema intitulado \textit{six honest serving men} para discutir as questões de elicitação de requisitos propostas por \textit{Laddaga} [185]. As seis questões são: Onde (está a causa da adaptação)?, Quando (adaptar)?, O quê (adaptar)?, Por que (motivação para adaptação)?, Quem (realizará a adaptação)? e Como (realizar a adaptação)?
Onde (está a causa da adaptação)? Essas perguntas encapsulam todas as questões relacionadas à identificação do problema que será tratado através de uma solução adaptativa (sejam problemas oriundos do ambiente ou decorrentes dos requisitos funcionais da aplicação).\\
Quando (adaptar)? Um dos grandes desafios envolvidos na modelagem de sistemas autogerenciaveis é entender em quais momentos a utilização de adaptação é necessária e viável. Em sistemas altamente dinâmicos a realização de adaptação sobre estados transientes do ambiente podem levar a situações de instabilidade [175]. Outro ponto a se observar é a frequência das operações de adaptação (adaptações muito frequêntes implicam \textit{overhead}, com consequente perda de desempenho e comprometimento do \textit{trade-off}).\\
O quê (adaptar)? A definição de quais partes do sistema computacional serão adaptadas varia. Sistemas autogerenciáveis administram desde novos valores de parâmetros, realizam a substituição/adição de componentes do sistema, chegando até a modificar estruturas arquiteturais. Geralmente demandam conhecimento de projeto e desempenho disponíveis em \textit{run-time}.\\
Por que (motivação para adaptação)? Essa questão administra os objetivos da adaptação tendo em vista o quanto os mecanismos \textit{self-adaptive} aproximam o sistema das metas estabelecidas. Pesquisas atuais investigam a prospecção e mapeamento de metas operacionais com foco em adaptação [305, 294].\\
Quem (realizará a adaptação)? Essa pergunta trata os diversos graus de automação que podem ser alcançados nos sistemas autogerenciados. Busca-se nesse momento entender os limites do envolvimento humano frente a automação. Ainda que em grande parte dos casos deseje-se o mínimo possível de intervenção humana nas rotinas de adaptação, em algumas situações torna-se desejável a inclusão de usuários nas ações de adaptação [93].\\
Como (realizar a adaptação)? Questão que investiga os meios pelos quais os artefatos serão modificados em \textit{runtime}. Normalmente são implementados através de ações elementares a serem executadas, suas ordens de execução e seus \textit{overheads} inseridos.

\subsubsection{Implementação}

Nesse ponto, a distinção entre duas partes principais dos sistemas autogerenciados se faz necessário. Um primeiro sistema, dito \textbf{gerenciado}, é o sistema responsável pela implementação da regra de negócio da aplicação. Esse sistema gerenciado sofrerá as ações da adaptação. Por outro lado, um segundo sistema é responsável por dotar o \textit{software} gerenciado de capacidades auto adaptativas. Esse segundo sistema, chamado de sistema \textbf{gerenciador}, é responsável por adicionar as capacidades de autogerenciamento ao sistema gerenciado.\\ 
Ainda que uma grande quantidade de mecanismos de autogerenciamento tenham sido propostos nos últimos anos, muitos deles fundamentam-se na utilização de \textit{loops} de adaptação. Sendo assim, um sistema gerenciador consiste fundamentalmente em uma implementação de \textit{loop} de adaptação. O nível de acoplamento experimentado entre os sistemas gerenciador e gerenciado é uma decisão de projeto. Uma primeira possiblidade consiste em implementar \textit{loops} de adaptação (sistema gerenciador) de forma externa ao sistema gerenciado (Figura 1). Uma outra abordagem consiste em implementar sistemas gerenciadores e gerenciado em uma única estrutura (sistema gerenciador faz parte do sistema gerenciado).\\
A seguir serão apresentadas as principais características dos loops de controle MAPE.\\
\textbf{Inserir figura dos sistemas gerenciadores e gerenciados...}\\
Os loops de controle MAPE foram inicialmente apresentadados pela IBM no artigo intitulado \textit{An architectural blueprint for autonomic computing}. Diversas discussões acerca dos \textit{loops MAPE-K} foram realizadas ao longo do tempo no contexto dos sistemas auto adaptativos [111]. Tais \textit{loops} fundamentam-se na execução, em sequencian das seguintes fases: \textbf{M}onitoramento, \textbf{A}nálise, \textbf{P}lanejamento e \textbf{E}xecução. A fase de monitoramento é responsável por obter informações acerta do estado do sistema. A atividade de análise consiste na avaliação das informações obtidas na fase de monitoramento. Após a avaliação de tais dados, a etapa de análise deverá deliberar pela necessidade ou não de realização de adaptação. A fase de planejamento é responsável pela definição de quais atividades de adaptação serão realizadas para que o sistema retorne a um estado que interesse à análise. Por fim, a atividade de execução consiste em realizar de forma efetiva as tarefas definidas pela fase de planejamento.\\
Uma série de variações dessa estrutura incial de \textit{loops} de adaptação são encontradas na literatura, a exemplo do MAPE-K (apresenta as mesmas fases dos \textit{loops MAPE}, adicionando uma base de conhecimento K).


\subsection{Padrões para Controle Descentralizado em Sistemas auto-adaptativos}
Quando os sistemas são heterogêneos, grandes e complexos um único \textit{loop} de controle pode não ser suficiente para gerenciar toda a adaptação [OnPatterns, ref 9,1]. Nesses casos, múltiplos \textit{loops} de controle podem ser empregados para gerenciar diferentes partes do sistema. Essa distribuição dos \textit{loops} de controle nos diversos nós do sistema traz uma complexidade a mais para o \textit{design }da solução: a coordenação na execução dos diversos \textit{loops}. Dessa problemática surge uma importante decisão a ser tomada em tempo de projeto: como se dará a cooperação entre os diversos \textit{loops} de controle. Nesse momento algumas questões surgem. Todas as fases dos controles MAPE (monitoração, análise, planejamento e execução) irão cooperar entre sí em todo o sistema? Cada sistema gerenciador realizará isoladamente a fase de monitoração (sem cooperação) e as demais fases serão realizadas em um único \textit{loop} de controle? A resposta para essas questões foi estudada no artigo \textit{On Patterns for decentralized Control in Self Adaptive Systems} [referenciaBLA].

De acordo com [referenciaBLA] diferentes padrões para implementação de \textit{loops} de controle tem sido utilizados de maneira prática na indústria. O \textit{Framework Rainbow}, por exemplo, distribui as fases de monitoração e execução sobre os diversos nós do sistema ao passo que mantem as fases de análise e planejamento centralizadas [referencia 17 do onPattern]. Por outro lado, os \textit{blue prints} arquiteturais da IBM [referencia 25 do onPattern] organizam os \textit{loops} MAPE hierarquicamente, mantendo em cada nível da hierarquia todas as quatro fases do \textit{loop} MAPE.\\
Em [referencia para onPattern] um conjunto de seis padrões para implementação de controle descentralizado é apresentado. Para cada padrão são apresentados os atributos de qualidade decorrentes da sua adoção, bem como as restrições impostas por tais decisões. Dentre os seis padrões apresentados em [referencia para onPattern] encontra-se o \textit{Regional Planning}. O padrão \textit{Regional Planning} aplica-se fundamentalmente em situações onde diversas partes integradas (regiões) de um sistema demandam não apenas por adaptações locais (dentro das regiões) mas também por adaptações globais (que extrapolam os limites das regiões). A solução proposta pelo \textit{Regional Planning} provê um único componente P (\textit{planning}) para cada região. Esse planejador regional é responsável por coletar as informações de todos os componentes que fazem parte da sua região para que o planejamento seja realizado. Planejadores regionais coordenam entre si (sobre múltiplas regiões) para realização do planejamento da adaptação global, ao passo que todos os outros componentes MAPE operam isoladamente.\\
A figura que segue ilustra uma instância do padrão \textit{Regional Planning}.\\
\textbf{Inserir figura do padrao...}\\
Na figura verifica-se a existência de dois diferentes tipos de nós de controle. O primeiro tipo de nó apresenta apenas o componente MAPE P. Em cada região existirá apenas um nó do tipo P. O segundo tipo de nó de controle apresenta os outros três componentes MAPE restantes: MAE. Em cada região existirão múltiplos nós de controle do tipo MAE. Em cada região os múltiplos componentes do tipo M  realizarão a monitoração do seu sistema local e do seu ambiente de execução. O componente A realizará uma análise local dos dados monitorados e os reportará para o planejador reginal P. O planejador regional P irã então planejar a adaptação local da região após cooperar com os planejadores P das outras regiões. Dessa forma a adaptação, em alguma medida, buscará ótimos globais, extrapolando os limites da monitoração e análise locais. Uma vez que os diversos planejadores regionais P concordem com um plano de adaptação os componentes E estarão prontos para ser executados localmente.\\
Como consequência da adoção do padrão \textit{Regional Planning} verifica-se uma redução na quantidade de dados trafegados uma vez que a monitoração e análise são realizadas localmente -- a análise local dos dados monitorados reduz a quantidade de dados e a frequência com que interações serão realizadas com o nó central da região. Por outro lado, a adoção do padrão \textit{Regional Planning} pode apresentar como \textit{overhead} a necessidade de agregar localmente no nó de planejamento os resultados das diversas operações de análise. Outra desvantagem é a necessidade de uma fase de planejamento demasiada detalhada, uma vez que as fases de execução não coordenam.



\section{Proposta}

Este trabalho tem como objetivo o projeto, implementação e avaliação de um modelo adaptativo para auto gerenciamento em sistemas \textit{self-adaptive} de larga escala. Esse modelo prevê a reconfiguração dinâmica da topologia do sistema gerenciador e da forma de comunicação entre os múltiplos componentes locais do controle. Através dessa abordagem, decisões de projeto acerca da topologia de controle ser adotada serão levadas para \textit{runtime}. Com isso, espera-se obter desempenho satisfatório do controle tanto em situações onde o sistema gerenciado/ambiente sugerem a adoção de topologias mais centralizadas, até situações onde padrões completamente descentralizados apresentam melhor \textit{trade-off}.

Tendo em vista que os sistemas gerenciadores atuais apresentam os mesmos desafios que motivaram a adoção da auto adaptação nos sistemas gerenciados, considerou-se a possibilidade de projetar controladores com características tais onde, eles mesmos, seriam vistos como sistemas auto adaptativos. Dessa forma, as decisões acerca de qual padrão seria adotado para implementação do sistema gerenciador não mais será uma decisão a ser tomada de maneira \textit{off-line}. Através do modelo proposto, a depender do seu estado interno e das condições do ambiente, a arquitetura do sistema gerenciador será reconfigurada. A arquitetura do sistema gerenciador irá se adaptar desde abordagens mais centralizadas até os extremos providos pelos padrões mais distribuídos.

Sabe-se que a adoção de controle implica \textit{overhead} [?]. Para verificar os possíveis ganhos de desempenho decorrentes da adoção do sistema controlador utilizando o modelo proposto, uma validação utilizando simulação de eventos discretos foi realizada. Através dessa validação busca-se verificar a eficiência do modelo proposto frente a adoção de padrões arquiteturais estáticos. A seção 5 apresenta os métodos e materiais utilizados na validação do modelo proposto. A seção 6 traz os resultados alcançados.

As seções que seguem apresentam em detalhes o modelo proposto.

\subsection{O Padrão \textit{Reginal Planning}}
Uma vez que o modelo proposto implementa uma variação do padrão \textit{Reginal Planning} serão apresentados a seguir os principais elementos que o caracterizam.

O padrão \textit{Reginal Planning} apresenta como principal característica a definição do conceito de Região. Para o \textit{Reginal Planning} pode-se entender Região como uma porção do sistema gerenciado a ser mantido por um único Planejador MAPE. Sabe-se que cada região apresentará dentre os seus diversos nós um conjunto de componentes MAPE do tipo MAE, mas somente um componente do tipo P.

Utilizando-se como exemplo um \textit{cluster Hadoop} formado por 40.000 máquinas, ao supor que o tamanho das regiões é fixo e igual a 400 teremos 100 regiões. De acordo com o padrão \textit{Reginal Planning} teremos necessariamente 100 nós com componentes MAPE do tipo P (um para cada região) e 40.000 componentes MAPE do tipo MAE (um para cada nó do \textit{cluster}). Observa-se nesse exemplo que o tamanho das regiões (400) foi uma decisão de projeto. De maneira \textit{off-line} o arquiteto de software necessariamente deliberará acerca do tamanho das regiões. O padrão \textit{Reginal Planning} considera que uma configuração em tempo de projeto será realizada de forma a se definir os limites de cada uma das regiões.

Conforme discutido na seção de fundamentação teórica, a adoção do padrão \textit{Reginal Planning} apresenta um \textit{trade-off} que envolve a qualidade da adaptação e a quantidade de dados trafegados (e localmente mantidos) nos nós responsáveis pelo planejamento. É importante ratificar nesse ponto o caráter fundamental do tamanho das regiões nesse \textit{trade-off}. Quanto maior o tamanho das regiões maior será o custo computacional envolvido (quantidade de dados trafegados e localmente mantidos nos nós responsáveis pelo planejamento). Em contrapartida melhor será a qualidade de adaptação, uma vez que as informações de análise disponibilizadas para o planejador terão aspectos mais globais. Por outro lado, quanto menor for o tamanho das regiões menor será o \textit{overhead} envolvido, mas menor será também a qualidade da adaptação uma vez que o planejamento atuará com informações de análise com características locais.

Sendo assim, como característica fundamental do padrão \textit{Reginal Planning}, pode-se destacar os limites existentes entre as diversas regiões que compõe o sistema. Reconhecer os limites das diversas regiões e defini-los é atribuição do arquiteto de \textit{software}. É uma atividade que é realizada \textit{off-line}, em tempo de projeto, usualmente utilizando como critério para definição do tamanho mínimo das regiões as situações de pior caso [referencia].

\subsection{O Planejamento Regional Adaptativo Proposto}

A seguinte hipótese será avaliada nesse trabalho: "A utilização de sistemas gerenciadores com arquiteturas adaptativas melhora o \textit{trade-off} entre atendimento de metas globais e \textit{overhead} de controle.".

Conforme discutido na seção anterior, existe um inerente \textit{trade-off} que envolve a definição dos limites das regiões no padrão \textit{Reginal Planning}: quanto maior o tamanho das regiões maior será o custo computacional envolvido, mas em contrapartida melhor será a qualidade de adaptação; por outro lado, quanto menor for o tamanho das regiões menor será o \textit{overhead} envolvido, mas menor será também a qualidade da adaptação.




\section{Introduction}
% no \IEEEPARstart
This demo file is intended to serve as a ``starter file''
for IEEE conference papers produced under \LaTeX\ using
IEEEtran.cls version 1.8b and later.
% You must have at least 2 lines in the paragraph with the drop letter
% (should never be an issue)
I wish you the best of success.

\hfill mds
 
\hfill August 26, 2015

\subsection{Subsection Heading Here}
Subsection text here.


\subsubsection{Subsubsection Heading Here}
Subsubsection text here.


% An example of a floating figure using the graphicx package.
% Note that \label must occur AFTER (or within) \caption.
% For figures, \caption should occur after the \includegraphics.
% Note that IEEEtran v1.7 and later has special internal code that
% is designed to preserve the operation of \label within \caption
% even when the captionsoff option is in effect. However, because
% of issues like this, it may be the safest practice to put all your
% \label just after \caption rather than within \caption{}.
%
% Reminder: the "draftcls" or "draftclsnofoot", not "draft", class
% option should be used if it is desired that the figures are to be
% displayed while in draft mode.
%
%\begin{figure}[!t]
%\centering
%\includegraphics[width=2.5in]{myfigure}
% where an .eps filename suffix will be assumed under latex, 
% and a .pdf suffix will be assumed for pdflatex; or what has been declared
% via \DeclareGraphicsExtensions.
%\caption{Simulation results for the network.}
%\label{fig_sim}
%\end{figure}

% Note that the IEEE typically puts floats only at the top, even when this
% results in a large percentage of a column being occupied by floats.


% An example of a double column floating figure using two subfigures.
% (The subfig.sty package must be loaded for this to work.)
% The subfigure \label commands are set within each subfloat command,
% and the \label for the overall figure must come after \caption.
% \hfil is used as a separator to get equal spacing.
% Watch out that the combined width of all the subfigures on a 
% line do not exceed the text width or a line break will occur.
%
%\begin{figure*}[!t]
%\centering
%\subfloat[Case I]{\includegraphics[width=2.5in]{box}%
%\label{fig_first_case}}
%\hfil
%\subfloat[Case II]{\includegraphics[width=2.5in]{box}%
%\label{fig_second_case}}
%\caption{Simulation results for the network.}
%\label{fig_sim}
%\end{figure*}
%
% Note that often IEEE papers with subfigures do not employ subfigure
% captions (using the optional argument to \subfloat[]), but instead will
% reference/describe all of them (a), (b), etc., within the main caption.
% Be aware that for subfig.sty to generate the (a), (b), etc., subfigure
% labels, the optional argument to \subfloat must be present. If a
% subcaption is not desired, just leave its contents blank,
% e.g., \subfloat[].


% An example of a floating table. Note that, for IEEE style tables, the
% \caption command should come BEFORE the table and, given that table
% captions serve much like titles, are usually capitalized except for words
% such as a, an, and, as, at, but, by, for, in, nor, of, on, or, the, to
% and up, which are usually not capitalized unless they are the first or
% last word of the caption. Table text will default to \footnotesize as
% the IEEE normally uses this smaller font for tables.
% The \label must come after \caption as always.
%
%\begin{table}[!t]
%% increase table row spacing, adjust to taste
%\renewcommand{\arraystretch}{1.3}
% if using array.sty, it might be a good idea to tweak the value of
% \extrarowheight as needed to properly center the text within the cells
%\caption{An Example of a Table}
%\label{table_example}
%\centering
%% Some packages, such as MDW tools, offer better commands for making tables
%% than the plain LaTeX2e tabular which is used here.
%\begin{tabular}{|c||c|}
%\hline
%One & Two\\
%\hline
%Three & Four\\
%\hline
%\end{tabular}
%\end{table}


% Note that the IEEE does not put floats in the very first column
% - or typically anywhere on the first page for that matter. Also,
% in-text middle ("here") positioning is typically not used, but it
% is allowed and encouraged for Computer Society conferences (but
% not Computer Society journals). Most IEEE journals/conferences use
% top floats exclusively. 
% Note that, LaTeX2e, unlike IEEE journals/conferences, places
% footnotes above bottom floats. This can be corrected via the
% \fnbelowfloat command of the stfloats package.




\section{Conclusion}
The conclusion goes here. Example citing: \cite{IEEEexample:articleetal}.




% conference papers do not normally have an appendix


% use section* for acknowledgment
\section*{Acknowledgment}


The authors would like to thank...





% trigger a \newpage just before the given reference
% number - used to balance the columns on the last page
% adjust value as needed - may need to be readjusted if
% the document is modified later
%\IEEEtriggeratref{8}
% The "triggered" command can be changed if desired:
%\IEEEtriggercmd{\enlargethispage{-5in}}

% references section

\bibliographystyle{IEEEtran}
\bibliography{IEEEabrv,ecdu-eudes}



% that's all folks
\end{document}


